\section{PROBLEMA} 
El siguiente proyecto se enmarca en el contexto del repositorio de la Universidad Privada de Tacna,
el cual se caracteriza por cumplir con la necesidad de implementación del servicio de repositorio virtual.
Para representar en este documento el diseño de distintos  diagramas de UML para evaluar los distintos procesos del repositorio para la
comunidad universitaria o externa, para mejorar
las condiciones de búsqueda y acceso a la información por parte de estudiantes y docentes.

\vspace*{0.3in}
En la primera parte del  documento se desarrollará el analisis de los diferentes diagramas de UML de los procesos 
como el acceso a la información, etc.
Posteriormente, se presentará el digrama de clases y el modelo de entidad - relacion que permite conocer cuáles son los
objetos que interactuan en el repositorio.

\vspace*{0.3in}
Finalmente se mostrarán los metodos realizados para evaluar su funcionamiento en las pruebas unitarias realizadas en Visual Studio, el modelo en el que se representan los procedimientos y
actividades del repositorio  para que la Institución los tome como herramientas en el momento que
se realice su implementación y así potencie los servicios que esta biblioteca puede
llegar a ofrecer a sus usuarios